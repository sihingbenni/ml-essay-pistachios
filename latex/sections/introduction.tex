
\section{Einleitung}

- Hier wird das Thema eingeführt, die Problemstellung erläutert 
- Zielsetzung der Arbeit wird hier klar formuliert

- Zudem wird i. d. R. der Aufbau der Arbeit kurz umrissen.
Auch hier schon passive Formulierungen:
- „…


In dieser Arbeit wird die Anwendung von K-Nächster-Nachbarn (K-NN)-Verfahren und Support-Vektor-Maschinen (SVM) für die Klassifizierung zweier Pistazienarten untersucht. 

% Zielsetzung
Ziel ist es den Einfluss an Hyperparametern 

% Aufbau der Arbeit
\subsection{Aufbau}
Zunächst wird der vorliegende Datensatz eingeführt, genauer analysiert, aufbereitet und in Trainings- sowie Testdaten aufgeteilt.
Sobald der Datensatz gründlich vorbereitet wurde wird mit der Klassifikation mittels K-NN Verfahren begonnen. 
Es wird mit unterschiedlichen \glqq{}K\grqq{} und Anzahl an Attributen trainiert. 
Die Ergebnisse werden unter anderem auf Genauigkeit und F1-Score anhand der Testdaten mit einander verglichen.


1. Klassifikation mit K-NN
verfügbare Hyperparameter:
 k - Die Anzahl an zu betrachtenden Nachbarn
 attribute - 16 zur Verfügung stehende
Es wurde eine Klasse erstellt welche es erleichtert verschiedene Kombinationen an Attributen und Ks auf ihre Genauigkeit zu testen. 
