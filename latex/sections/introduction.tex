
\section{Einleitung}

- Hier wird das Thema eingeführt, die Problemstellung erläutert 
- Zielsetzung der Arbeit wird hier klar formuliert

- Zudem wird i. d. R. der Aufbau der Arbeit kurz umrissen.
Auch hier schon passive Formulierungen:
- „…


In dieser Arbeit wird die Anwendung von K-Nächster-Nachbarn (K-NN)-Verfahren und Support-Vektor-Maschinen (SVM) für die Klassifizierung zweier Pistazienarten untersucht. 

\subsection{Zielsetzung}

\subsection{Aufbau}
Zunächst wird der vorliegende Datensatz eingeführt, genauer analysiert, aufbereitet und in Trainings- sowie Testdaten aufgeteilt.
Sobald dies geschehen ist kann es mit der Klassifikation los gehen.

1. Klassifikation mit K-NN
verfügbare Hyperparameter:
 k - Die Anzahl an zu betrachtenden Nachbarn
 attribute - 16 zur Verfügung stehende
Es wurde eine Klasse erstellt welche es erleichtert verschiedene Kombinationen an 
