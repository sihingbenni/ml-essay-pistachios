
\section{Einleitung}
In dieser Arbeit wird die Anwendung von K-Nächster-Nachbarn (K-NN)-Verfahren und Support-Vektor-Maschinen (SVM) für die Klassifizierung zweier Pistazienarten untersucht. 

% Zielsetzung
\subsection*{Zielsetzung}
Das vorrangige Ziel dieser Arbeit ist es, den Einfluss verschiedener Hyperparameter beider Klassifikationsmethoden zu untersuchen und zu bewerten, wie diese die Leistungsfähigkeit der Modelle beeinflussen. Dabei liegt der Fokus auf der Identifikation der optimalen Konfiguration der Hyperparameter für K-NN und SVM, um eine möglichst hohe Genauigkeit und einen hohen F1-Score bei der Klassifikation der Pistazienarten zu erreichen.

% Aufbau der Arbeit
\subsection*{Aufbau der Arbeit}
Der Aufbau dieser Arbeit ist wie folgt gegliedert:
\begin{itemize}
	\item \textbf{Einführung in den Datensatz:} Zuerst wird der für diese Untersuchung verwendete Datensatz vorgestellt. Dies schließt eine detaillierte Analyse und Aufbereitung des Datensatzes ein, gefolgt von der Aufteilung in Trainings- und Testdatensätze.
	\item \textbf{Anwendung von K-NN:} Nach der Datenvorbereitung wird der Klassifikationsprozess mittels K-NN durchgeführt. Dabei werden verschiedene Anzahlen von \glqq{}K\grqq{} sowie die Auswahl der Attribute experimentell variiert, um deren Auswirkungen auf die Modellleistung zu untersuchen.
	\item \textbf{Anwendung von SVM:} Im Anschluss an K-NN wird eine ähnliche Untersuchung mit SVM durchgeführt. Hier wird eine Vielzahl an Hyperparametern betrachtet und die Auswirkungen dieser Variationen auf die Klassifikationsleistung beleuchtet.
	\item \textbf{Vergleich und Auswertung:} Die trainierten Klassifikatoren werden basierend auf ihrer Leistung auf den Testdaten bewertet. Die Bewertung erfolgt dabei durch den Vergleich von Genauigkeit und F1-Score, um die Effektivität der verschiedenen Hyperparameter-Konfigurationen zu erörtern.
	\item \textbf{Schlussfolgerung und Ausblick:} Abschließend werden die gewonnenen Erkenntnisse zusammengefasst und diskutiert. Zudem wird ein Ausblick auf mögliche weitere Optimierungsansätze gegeben.
\end{itemize}

Durch diesen strukturierten Ansatz wird es möglich, einen tiefgehenden Einblick in die Eignung und Optimierung von K-NN und SVM für die spezifische Anwendung der Pistazienarten-Klassifikation zu gewähren.

