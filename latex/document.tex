% !TeX document-id = {9f30a0a7-fd28-429c-b30f-1f1ceab3f7e5}
% tell Tex Studio to compile with --shell-escape to visualize SVGs
% !TeX TXS-program:compile = txs:///pdflatex/[--shell-escape]

\pdfsuppresswarningpagegroup=1

%\documentclass[11pt, a4paper, twoside=true, ngerman, parskip=full, listof=totoc,footsepline]{scrartcl} 
\documentclass[
11pt, 
a4paper,
twoside=false,
headings=normal, 
ngerman, 
parskip=full, 
listof=totoc, 
numbers=noenddot,
toc = bibnumbered	
]{scrartcl} 

% geschwungene Klammer Pflicht
% eckige Klammer optionaler Parameter
% article book report letter
% scrartcl scrreprt scrbook scrlettr2

% listof totoc alle Verzeichnisse werden ins Inhaltsverzeichnis aufgenommen

% standardpakete die immer sinnvoll sind
\usepackage[utf8]{inputenc}
\usepackage[ngerman]{babel}
\usepackage{csquotes}
\usepackage{abstract}
% Schriftart wechseln
% LateX font catalogue
% use arial like clon
%\usepackage{arev}
%\usepackage[T1]{fontenc}

% Benutzen für Einheiten
\usepackage{siunitx}
\sisetup{locale=DE, per-mode=fraction, separate-uncertainty}

\usepackage{graphicx,amsmath,amssymb,amsfonts}

% nutze verbesserte schriftart von lateX
\usepackage{microtype}

\usepackage{todonotes}

% blindtext
\usepackage{blindtext}

% svg graphics
\usepackage[inkscapearea=page]{svg}

% microtype verbessert den Blocksatz
\usepackage{microtype}


% Ränder konfigurieren
\usepackage[
top=2.5cm, % Top margin
bottom=2.5cm, % Bottom margin
left=2cm, % Left margin
right=2cm, % Right margin
footskip=1cm, % Space from the bottom margin to the baseline of the footer
headsep=0.75cm, % Space from the top margin to the baseline of the header
columnsep=20pt, % Space between text columns (in twocolumn mode)
%showframe % Uncomment to show frames around the margins for debugging purposes
]{geometry}

% empty = keine Seitenzahl kein nichts
\pagestyle{empty}

% plain = nur seitenzahl
\pagestyle{plain}

\usepackage{scrlayer-scrpage}
\pagestyle{scrheadings}
\usepackage{multicol} % used for making multiple columns

\newcommand{\me}{Benjamin Puhani}

% command für Vergleiche.
\newcommand{\vgl}[2]{\autocite[Vgl.][S.~2]{#1}}
% Nutzung \vgl{Literatur-ID}{Seitenangabe}

% command to bold and kursive
\newcommand{\boldIt}[1]{\textbf{\textit{#1}}}

% eine neue Environment erstellen
\newenvironment{beispiel}{\begin{center}\itshape\bfseries}{\end{center}}

%\automark{chapter}
\ihead[]{\headmark}
\chead[]{\me}
\ohead[\today]{\today}
\cfoot[Maschinelles Lernen]{Maschinelles Lernen}
\ofoot[\pagemark]{\pagemark}


% Zeilenabstand ändern
\usepackage[onehalfspacing]{setspace}

\onehalfspacing

% Erstelle Befehl um eine Neue Equation dem Verzeichnis hinzuzufügen
\newcommand{\myequations}[1]{\addcontentsline{equ}{equation}{\protect\numberline{\theequation}~~#1}}

\usepackage{newfloat}
\DeclareFloatingEnvironment[fileext=loa, name=Gleichung]{gleichung}

% Erstelle einen Math operator
\DeclareMathOperator{\sincos}{}

% Tabellen mit booktabs
% \toprule
% \midrule
% \bottomrule
\usepackage{booktabs}

% verwendet für die zitation
\usepackage[backend=biber,style=ieee,bibwarn=true,bibencoding=utf8]{biblatex}
% bibdatei hinzufügen
\addbibresource{bibliography/literature.bib}

%\usepackage{pgfplot}

% muss als vorletztes Paket geladen werden
\usepackage[]{hyperref}

% Paket für Referencen
% muss letzter Import eines Paketes sein.
% Erkennt was es ist und ist besser als normales LateX
\usepackage[noabbrev]{cleveref}

% Auskommentieren mit Strg + T

\begin{document}

	\begin{titlepage}
	\begin{center}
		\begin{figure}
			\centering
			\includesvg[width=7cm]{images/fachhochschule-kiel-logo.svg}
		\end{figure}
		Fachhochschule Kiel\\
		Fachbereich Informatik und Elektrotechnik\\
		Studiengang Informatik (B.Sc)\\
		Maschinelles Lernen\\
		Herr Prof. Dr. Patrick Hennig\\
		WiSe 2024/25\\
		
	\end{center}
	\vfill
	\begin{center}
		\Huge \textsc{\mytitle}
		
		
		\vfill
		
		\LARGE
		Hausarbeit
		
		\vspace{1cm}
		vorgelegt von\\
		Benjamin Puhani\\
		Matr. Nr.: 941077\\
		am \today{}
	\end{center}
	\vfill
	
	
	\begin{tabular}{@{}l@{\hspace{2cm}}l@{}}
		Erstprüfer:   & Herr Prof. Dr. Patrick Hennig\\
	\end{tabular}

\end{titlepage}



%Titel der Arbeit,
%Zweck der Arbeit (Thesis zur Erlangung des Grades Bachelor / Master of Science /
%Engineering),
%Organisation (Fachhochschule Kiel, Fachbereich Informatik und Elektrotechnik),
%Autor (euer Name),
%Abgabedatum,
%Matrikelnummer,
%Namen von Erst- und Zweitprüfer,
%ggf. Name des Betreuers in der Firma,
%ggf. Name und Anschrift der Firma,
%evtl. das Logo der FH-Kiel
	\section*{Eigenständigkeitserklärung}
\addcontentsline{toc}{section}{Eigenständigkeitserklärung}
Hiermit versichere ich an Eides statt, dass ich die vorliegende Arbeit selbstständig und ohne Benutzung anderer als der in den Fußnoten und im Literaturverzeichnis angegebenen Quellen angefertigt habe.

Kiel den \today

\vspace{1cm} % Abstand vor der Unterschriftlinie

\rule{6cm}{1pt} % Zeichnet eine Linie (Breite x Höhe)
\\ % Erzwingt den Zeilenumbruch
\me % Der Name unter der Linie\pagenumbering{Roman}
	\section*{KI-Erklärung}
\addcontentsline{toc}{section}{KI-Erklärung}

\todo[inline]{Überarbeiten}

Hiermit versichere ich an Eides statt, dass

%%%%

In dieser wissenschaftlichen Arbeit wurden Künstliche-Intelligenz (KI)-Technologien zur Unterstützung verschiedener Aspekte der Forschung eingesetzt. Die Nutzung umfasste unter anderem die Analyse und Auswertung von Literatur, die Unterstützung bei der Datenauswertung sowie die Generierung von Ideen und Inhalten.

Es wird ausdrücklich darauf hingewiesen, dass die endgültige Verantwortung für die inhaltliche Richtigkeit, die kritische Reflexion und die Interpretation der Ergebnisse beim Autor/der Autorin dieser Arbeit liegt.

Die KI diente lediglich als Werkzeug und nicht als Ersatz für das kritische und analytische Denken des Forschenden.

%%%%

Kiel den \today

\vspace{1cm} % Abstand vor der Unterschriftlinie

\rule{6cm}{1pt} % Zeichnet eine Linie (Breite x Höhe)
\\ % Erzwingt den Zeilenumbruch
\me % Der Name unter der Linie
	\newpage
	\begin{abstract}
Diese Arbeit entstand als Antwort auf die spezifische Aufgabenstellung, mindestens zwei Klassifikationsmethoden auf einen Datensatz anzuwenden, der mindestens 200 Elemente umfasst, Klassen aufweist, die nicht linear separierbar sind, und einen Merkmalsraum mit mindestens zwei Dimensionen besitzt. In dieser Studie wurden das k-Nächster-Nachbarn (k-NN)-Verfahren und Support-Vektor-Maschinen (SVM) zur Klassifikation zweier Pistazienarten verwendet. Die Untersuchung folgte einem strukturierten Ansatz, der eine explorative Datenanalyse, Datenbereinigung, Aufteilung in Trainings- und Testdaten, eine systematische Variation von Hyperparametern sowie deren Bewertung mit geeigneten Qualitätsmaßen umfasste. Die signifikante Verbesserung der Klassifikationsgenauigkeit durch Hyperparameter-Optimierung und Merkmalsnormalisierung demonstriert die Effektivität beider Methoden in diesem spezifischen Anwendungsfeld. Besondere Aufmerksamkeit wurde dem Einfluss von Hyperparametereinstellungen auf die Leistungsfähigkeit der Modelle gewidmet, wobei für k-NN durch die Variation des 'k'-Wertes und die Auswahl von sechs Merkmalen eine optimale Genauigkeit von 89,64 \% erreicht wurde. Im Vergleich dazu wurde durch die Anwendung eines systematischen \glqq{}GridSearch\grqq{}-Verfahrens zur Optimierung der Hyperparameter für SVM eine ähnlich hohe Leistung erzielt.
Die Ergebnisse bieten Einblicke in die Potenziale und Herausforderungen bei der Anwendung von k-NN und SVM in der nicht-linearen Klassifikation und unterstreichen die Bedeutung sorgfältiger Datenanalyse und -aufbereitung für die Modellgenauigkeit. Zukünftige Forschungen könnten sich auf die Vertiefung der Untersuchung von Overfitting konzentrieren, um die Generalisierbarkeit der Modelle zu verbessern. 

%Diese Studie bietet einen Überblick über die Anwendung des k-NN-Verfahrens zur Klassifikation von Daten unter besonderer Berücksichtigung von Skalierungs- und Normalisierungstechniken.
%Aufgrund des begrenzten Umfangs dieser Arbeit werden detaillierte Analysen, einschließlich umfassender Datenvisualisierungen und Codebeispiele, in einem ergänzenden Jupyter Notebook bereitgestellt, das für eine vertiefte Betrachtung herangezogen werden kann.
\end{abstract}
\pagenumbering{arabic}
	
	\begin{multicols}{2}
		
\section{Einleitung}

- Hier wird das Thema eingeführt, die Problemstellung erläutert 
- Zielsetzung der Arbeit wird hier klar formuliert
- In wissenschaftlichen „Papern“  findet sich hier Einordnung der 
Fragestellung in wissenschaftliche Vorarbeiten -> Literaturarbeit 
- Zudem wird i. d. R. der Aufbau der Arbeit kurz umrissen.
Auch hier schon passive Formulierungen:
- „…
In der vorliegenden Ausarbeitung wird ein Nächster-Nachbar-Verfahren 
zur Klassifizierung von Witterungs- und Schneeverhältnissen in 
Lavinenrisikoklassen eingesetzt.

\subsection{Zielsetzung}


		
\chapter{Kernprinzipien und Anwendungen (Hauptteil)}

Dies ist der Hauptteil der Arbeit

\section{Aufgabe: Literaturverzeichnis}
Damit im Literaturverzeichnis Einträge angezeigt werden können müssen diese dafür in dem Dokument erwähnt werden.
\\
Laut Aufgabe soll ein Paper\cite{Shute.2024}, eine Statistik\cite{Statista.08.10.2024} und eine Norm\cite{DIN-EN-ISO-9241-11.} eingebunden werden.


\section{Bilder, Tabellen \& Formeln}
\label{sect:bilder}

\begin{figure}[htbp]
	\centering
	\includesvg[width=7cm]{images/Schleswig-Holstein.svg}
	\caption{Das Logo des Landes Schleswig-Holstein}
	\label{fig:land-sh}
\end{figure}

\vfill

\begin{table}[htbp]
	\centering
	\caption{Physikalische Größen}
	\begin{tabular}{lc}
		\toprule
		Physikalische Größe & SI-Einheit \\
		\midrule
		Drehmoment & $Nm$ \\
		Leistung & Watt \\
		Fläche & $m^2$ \\
		Spannung/Druck & Pascal \\
		Ruck & $\frac{m}{s^3}$ \\
		\bottomrule
	\end{tabular}
	\label{tab:physikalische_größen}
\end{table}


\begin{gleichung}[b]
	\centering
	\begin{equation}
		E = mc^2
	\label{eq:emc2}
	\end{equation}
	\caption{Die Formel der Energie-Masse-Äquivalenz}
\end{gleichung}

\newpage

\begin{itemize}
	\item Hier soll zum einen auf die Grafik \cref{fig:land-sh} verlinkt werden.
	\item Zum anderen auf eine Tabelle: \cref{tab:physikalische_größen}
	\item Und des weiteren auf eine Formel: \cref{eq:emc2}
\end{itemize}



\chapter{Den Roten Faden nicht aus den Augen verlieren}
\label{chapt:roterFaden}

Wichtig ist es bei all den Grafiken, Tabellen und Gleichungen aus dem Kapitel \ref{sect:bilder} nicht den Überblick und noch viel wichtiger den Roten Faden zu verlieren. Deshalb wird hier nun noch einmal hervorgehoben was das Land SH mit Physikalischen Größen und mit Albert Einstein zu tun hat. 


\chapter{Dennoch zum Ende kommen}
\label{chap:zumEndeKommen}

Wie das vorherige Kapitel \ref{chapt:roterFaden} gezeigt hat, ist die Fähigkeit, den Roten Faden nicht zu verlieren, essentiell für das Verständnis der Verbindung zwischen scheinbar unzusammenhängenden Elementen - wie dem Land Schleswig-Holstein, physikalischen Größen und Albert Einstein. 


Jetzt, da diese Verbindungen klar aufgezeigt und diskutiert wurden ist an der Zeit, die gewonnenen Erkenntnisse zusammenzutragen.


\chapter{Ergebnisse}
\label{chap:ergebnisse}

Ergebnisse gibt es viele. Doch am wichtigsten sind die Erkenntnisse aus \ref{sect:bilder}. 

Und, dass in \LaTeX{} mittels:

\begin{verbatim}
	\chapter{Wichtiges Kapitel}
	\label{chap:wichtiges_kap}
	
	dann mit:
	\ref{chap:wichtiges_kap}
\end{verbatim}

verlinkt werden kann.

\section{Größe des PDFs}

Das PDF ist derzeit \textbf{244 KB} groß und damit klein genug um per Email versendet zu werden.


		\section{Diskussion}

- Diskussion der Ergebnisse inkl. eines Vergleichs der Methoden
		
		
		\printbibliography[heading=bibintoc]
	\end{multicols}
	\listoftodos[Notes] % todo remove!!
\end{document}
