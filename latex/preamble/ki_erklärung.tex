\section*{KI-Erklärung}
\addcontentsline{toc}{section}{KI-Erklärung}

In dieser Arbeit wurden KI-basierte Werkzeuge genutzt, um sowohl die Formulierung der Texte als auch die Programmierung zu unterstützen. HAWKI\footnote{\url{https://hawki.fh-kiel.de/}} diente als Assistenzsystem für Formulierungshilfen, um die Klarheit und Struktur des Textes zu verbessern. Dabei blieb die inhaltliche Verantwortung und Ausarbeitung stets beim Autor. Für spezifische Programmieraufgaben, insbesondere bei der Erstellung von Diagrammen mittels Matplotlib, wurde GitHub Copilot eingesetzt. Dieser fungierte als Programmierassistent, der durch Vorschläge zur Codeerstellung die Implementierung effizienter gestaltete, ohne jedoch die konzeptionelle Entscheidungshoheit des Autors zu beeinträchtigen. Die Anwendung beider KI-Tools erfolgte mit dem Ziel, die Qualität der Arbeit zu steigern, wobei die intellektuelle und kreative Eigenleistung des Autors die Grundlage aller Inhalte bildet.

Kiel den \today

\vspace{1cm} % Abstand vor der Unterschriftlinie

\rule{6cm}{1pt} % Zeichnet eine Linie (Breite x Höhe)
\\ % Erzwingt den Zeilenumbruch
\me % Der Name unter der Linie